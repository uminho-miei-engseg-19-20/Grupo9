% Introdução
\section{Introdução}

\par Neste trabalho seram abordados 2 programas do OWASP (Open Web Application Security Project), comunidade online esta que cria e disponibiliza de forma gratuita artigos, metodologias, documentação, ferramentas e tecnologias no campo da segurança de aplicações web, posto isto, as ferrametas que seram analisadas e referidas neste relatório são então o OWASP Code Pulse e o OWASP ZAP.

\par O projeto OWASP Code Pulse é uma ferramenta que fornece informações sobre a cobertura de código em tempo real das atividades de teste da caixa preta, e caracteriza-se por ser uma aplicação desktop multiplataforma que funciona na maioria das plataformas principais. Esta ferramenta monitoriza o tempo de execução da aplicação de destino usando uma abordagem baseada em agente. O Code Pulse tem ainda a particularidade de oferecer suporte a programas Java e .NET Framework, para além de poder ainda rastrear os detalhes da cobertura de código ao nível do método ou do código-fonte para mostrar o que está a ser chamado e quando. 

\par O outro programa que será estudado também é o OWASP Zed Attack Proxy (ZAP), ferramenta esta que permite a realização de testes de penetração de código aberto. Esta ferramenta foi projetada especificamente para testar aplicações da Web e é flexível e extensível. O ZAP é o que é conhecido como um "proxy intermediário", ou seja, este fica entre o navegador da pessoa que está efetuar os testes e a aplicação Web, para que essa pessoa possa interceptar e inspecionar as mensagens enviadas entre o navegador e o aplicação Web, modificar o conteúdo, se necessário, e encaminhar esses pacotes para o destino. O ZAP destaca-se pelo facto de fornecer funcionalidades para diversos níveis de habilidade, ou seja, especialistas em testes de segurança e/ou novatos, tem ainda a particularidade de possuir versões para cada sistema operacional.
