%
\title{MSTG - Teste de aplicações Android e iOS}
%
%\titlerunning{Abbreviated paper title}
% If the paper title is too long for the running head, you can set
% an abbreviated paper title here
%
%\author{Adriana Lopes \and Diana Carrilho \and Henrique Faria \and Paulo Barbosa}
\author{}
%
% First names are abbreviated in the running head.
% If there are more than two authors, 'et al.' is used.
%
\institute{Departamento de Informática, Universidade do Minho}
%
\maketitle              % typeset the header of the contribution
%
% Abstract
\begin{abstract}

\keywords{MSTG  \and Android \and iOS.}
\end{abstract}
%
%
% Introdução
\begin{center}
\normalsize{\bfseries Introdução}\hfill 

O documento que se usa como base o MSTG explica o processo de testes de segurança para aplicações moveis. Apresentando conceitos gerais desde os tipos de testes que existem, como incorporar num ciclo de desenvolvimento passando por como testar e resolver problemas das aplicações moveis.

Ainda descreve os dois sistemas operativos principais dos sistemas moveis Android e iOS, como os seus prolemas de segurança conhecidos e como testa-los. 

O seguinte documento pretende apresentar de uma forma mais sucinta como esse processo é feito, dando alguns exemplos da metodologia usada pelo MSTG.
\end{center}






